\chapter{Introducción}
%%---------------------------------------------------------
La introducción del TFG debe servir para que los profesores que evalúan el Trabajo puedan comprender el contexto en el que se realiza el mismo, y los objetivos que se plantean.

Esta plantilla muestra la estructura básica de la memoria final de TFG, así como algunas instrucciones de formato.

El esquema básico de una memoria final de TFG es el siguiente:
\begin{itemize}
\item[•] Resumen en español e inglés (máximo 2 páginas cada uno)
\item[•] Tabla de contenidos
\item[•] Introducción (con los objetivos del TFG)
\item[•] Desarrollo
\item[•] Resultados y conclusiones
\item[•] Análisis de impacto
\item[•] Bibliografía (publicaciones utilizadas en el estudio y desarrollo del trabajo)
\item[•] Anexos (opcional)
\end{itemize}

%%---------------------------------------------------------

\section{Motivación}
La Topología se centra en el estudio de las diversas propiedades de los espacios topológicos y las funciones continuas. Mientras que en el subcampo de la Topología Computacional veremos cómo podemos hacer uso de diversos algoritmos para poder estudiar las propiedades de los espacios topológicos y ser capaces de resolver problemas topológicos computacionalmente.

\section{Descripción general del trabajo}
El trabajo se basa en el estudio y exposición del Teorema de estabilidad, el cual, a grandes rasgos, establece que pequeñas perturbaciones en los datos implican pequeñas perturbaciones en la homología persistente. Para ello me centraré en el artículo \cite{Cohen-Steiner2007}.

Adicionalmente, implementaré en Python la distancia \textit{Bottleneck} para poder ilustrar este teorema haciendo uso distintos conjuntos de datos. La implementación se sustenta en el uso de complejos simpliciales y su correspondiente homología simplicial.

\subsection*{Objetivos}
\begin{itemize}
	\item Buscar referencias que contengan el enunciado y la demostración del Teorema de estabilidad.
	\item Estudiar y entender estas referencias.
	\item Implementar el cálculo de la distancia de Bottleneck entre diagramas de persistencia.
	\item Ilustrar el Teorema de estabilidad sobre distintos conjuntos de datos.
\end{itemize}