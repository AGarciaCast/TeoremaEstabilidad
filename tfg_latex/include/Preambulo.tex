% -*-coding: utf-8 -*-
%%***********************************************
%% Plantilla para TFG.
%% Escuela Técnica Superior de Ingenieros Informáticos. UPM.
%%***********************************************
%% Preámbulo del documento.
%%***********************************************
\documentclass[a4paper,11pt]{book}
\usepackage[utf8]{inputenc}
\usepackage[T1]{fontenc}
\usepackage[english,spanish,es-lcroman]{babel}
\usepackage{bookman}
\decimalpoint
\usepackage{IEEEtrantools}
\usepackage{graphicx}
\usepackage{amsfonts,amsgen,amsmath,amssymb,amsthm}
\usepackage{mathtools,bm}
\usepackage{tensor}
\usepackage{faktor}
\usepackage{aligned-overset}
\usepackage{csvsimple}
\usepackage{longtable}

\DeclarePairedDelimiter\abs{\lvert}{\rvert}%
\DeclarePairedDelimiter\norm{\lVert}{\rVert}%

\makeatletter
\newtheorem*{rep@theorem}{\rep@title}
\newcommand{\newreptheorem}[2]{%
\newenvironment{rep#1}[1]{%
 \def\rep@title{#2 \ref{##1}}%
 \begin{rep@theorem}}%
 {\end{rep@theorem}}}
\makeatother


\theoremstyle{plain}% default
\newtheorem{theorem}{Teorema}[section]
\newreptheorem{theorem}{Teorema}
\newtheorem*{corollary}{Corolario}
\newtheorem{lemma}[theorem]{Lema}
\newreptheorem{lemma}{Lema}
\newtheorem{proposition}{Proposición}[section]

\theoremstyle{definition}
\newtheorem{definition}{Definición}[section]
\newtheorem{exmp}{Ejemplo}[section]
\newtheorem{property}{Propiedad}[section]

\theoremstyle{remark}
\newtheorem*{remark}{Observación}
\newtheorem*{note}{Nota}

\usepackage[top=3cm, bottom=3cm, right=2.54cm, left=2.54cm]{geometry}
\usepackage{afterpage}
\usepackage{colortbl,longtable}
\usepackage[pdfborder={0 0 0}]{hyperref} 
\usepackage{pdfpages}
\usepackage{url}
\usepackage[stable]{footmisc}
\usepackage{parskip} % para separar párrafos con espacio.

\usepackage{tikz}
\usetikzlibrary{babel}
\usetikzlibrary{arrows}
\usetikzlibrary{automata}
\usetikzlibrary{cd}

\usepackage{subcaption}

\definecolor{greeo}{RGB}{37,211,102}
\definecolor{redp}{RGB}{255,153,153}
\usetikzlibrary{patterns}

\usepackage{enumitem}
\renewcommand{\theenumi}{\Alph{enumi}}

\usepackage{todonotes}
\setuptodonotes{inline}

\newcommand{\bigO}{\mathcal{O}}
\newcommand{\cupdot}{\mathbin{\dot\cup}}
%%-----------------------------------------------
\usepackage{fancyhdr}
\pagestyle{fancy}
\fancyhf{}
\fancyhead[LO]{\leftmark}
\fancyhead[RE]{\rightmark}
\setlength{\headheight}{1.5\headheight}
\cfoot{\thepage}

\addto\captionsspanish{ \renewcommand{\contentsname}
  {Tabla de contenidos} }
\setcounter{tocdepth}{4}
\setcounter{secnumdepth}{4}

\renewcommand{\chaptermark}[1]{\markboth{\textbf{#1}}{}}
\renewcommand{\sectionmark}[1]{\markright{\textbf{\thesection. #1}}}
\newcommand{\HRule}{\rule{\linewidth}{0.5mm}}
\newcommand{\bigrule}{\titlerule[0.5mm]}

\usepackage{appendix}
\renewcommand{\appendixname}{Anexos}
\renewcommand{\appendixtocname}{Anexos}
%\renewcommand{\appendixpagename}{Anexos}
%%-----------------------------------------------
%% Páginas en blanco sin cabecera:
%%-----------------------------------------------
\usepackage{dcolumn}
\newcolumntype{.}{D{.}{\esperiod}{-1}}
\makeatletter
\addto\shorthandsspanish{\let\esperiod\es@period@code}

\def\clearpage{
  \ifvmode
    \ifnum \@dbltopnum =\m@ne
      \ifdim \pagetotal <\topskip
        \hbox{}
      \fi
    \fi
  \fi
  \newpage
  \thispagestyle{empty}
  \write\m@ne{}
  \vbox{}
  \penalty -\@Mi
}
\makeatother
%%-----------------------------------------------
%% Estilos código de lenguajes: Consola, C, C++ y Python
%%-----------------------------------------------
\usepackage{algorithm}
\usepackage{algpseudocode}
\floatname{algorithm}{Algoritmo}
\renewcommand{\algorithmicrequire}{\textbf{Entrada:}}
\renewcommand{\algorithmicensure}{\textbf{Salida:}}

\usepackage{color}
\usepackage{xcolor}
\definecolor{gray97}{gray}{.97}
\definecolor{gray75}{gray}{.75}
\definecolor{gray45}{gray}{.45}
\definecolor{lightgray}{rgb}{.9,.9,.9}

\usepackage{listings}
\renewcommand{\lstlistingname}{Anexo}
\lstset{ frame=Ltb,
     framerule=0pt,
     aboveskip=0.5cm,
     framextopmargin=3pt,
     framexbottommargin=3pt,
     framexleftmargin=0.1cm,
     framesep=0pt,
     rulesep=.4pt,
     backgroundcolor=\color{gray97},
     rulesepcolor=\color{black},
     %
     stringstyle=\ttfamily,
     showstringspaces = false,
     basicstyle=\scriptsize\ttfamily,
     commentstyle=\color{gray45},
     keywordstyle=\bfseries,
     %
     numbers=left,
     numbersep=6pt,
     numberstyle=\tiny,
     numberfirstline = false,
     breaklines=true,
     %
     extendedchars=true,
     literate=
  {á}{{\'a}}1 {é}{{\'e}}1 {í}{{\'i}}1 {ó}{{\'o}}1 {ú}{{\'u}}1
  {Á}{{\'A}}1 {É}{{\'E}}1 {Í}{{\'I}}1 {Ó}{{\'O}}1 {Ú}{{\'U}}1
  {à}{{\`a}}1 {è}{{\`e}}1 {ì}{{\`i}}1 {ò}{{\`o}}1 {ù}{{\`u}}1
  {À}{{\`A}}1 {È}{{\'E}}1 {Ì}{{\`I}}1 {Ò}{{\`O}}1 {Ù}{{\`U}}1
  {ä}{{\"a}}1 {ë}{{\"e}}1 {ï}{{\"i}}1 {ö}{{\"o}}1 {ü}{{\"u}}1
  {Ä}{{\"A}}1 {Ë}{{\"E}}1 {Ï}{{\"I}}1 {Ö}{{\"O}}1 {Ü}{{\"U}}1
  {â}{{\^a}}1 {ê}{{\^e}}1 {î}{{\^i}}1 {ô}{{\^o}}1 {û}{{\^u}}1
  {Â}{{\^A}}1 {Ê}{{\^E}}1 {Î}{{\^I}}1 {Ô}{{\^O}}1 {Û}{{\^U}}1
  {ã}{{\~a}}1 {ẽ}{{\~e}}1 {ĩ}{{\~i}}1 {õ}{{\~o}}1 {ũ}{{\~u}}1
  {Ã}{{\~A}}1 {Ẽ}{{\~E}}1 {Ĩ}{{\~I}}1 {Õ}{{\~O}}1 {Ũ}{{\~U}}1
  {œ}{{\oe}}1 {Œ}{{\OE}}1 {æ}{{\ae}}1 {Æ}{{\AE}}1 {ß}{{\ss}}1
  {ű}{{\H{u}}}1 {Ű}{{\H{U}}}1 {ő}{{\H{o}}}1 {Ő}{{\H{O}}}1
  {ç}{{\c c}}1 {Ç}{{\c C}}1 {ø}{{\o}}1 {å}{{\r a}}1 {Å}{{\r A}}1
  {€}{{\euro}}1 {£}{{\pounds}}1 {«}{{\guillemotleft}}1
  {»}{{\guillemotright}}1 {ñ}{{\~n}}1 {Ñ}{{\~N}}1 {¿}{{?`}}1 {¡}{{!`}}1 
   }
\lstnewenvironment{listing}[1][]
   {\lstset{#1}\pagebreak[0]}{\pagebreak[0]}

\lstdefinestyle{consola}{
   language=bash,
   stringstyle=\mdseries\rmfamily,
   keywordstyle=\bfseries\rmfamily,
   basicstyle=\scriptsize\bf\ttfamily,
   stepnumber=1,
   backgroundcolor=\color{lightgray},
   extendedchars=true,
   showstringspaces=false,
   showspaces=false,
   tabsize=2,
   breaklines=true,
   showtabs=false,
   captionpos=b,
   rangeprefix=-------------,
   rangesuffix=-------------,
   includerangemarker=false,
   columns=flexible,
   firstnumber=0
}

\newcommand{\consola}[1]{
\lstinputlisting[style=consola,linerange=INICIO:#1-FIN:#1]{../code/ejemplos.log}
}


\lstdefinestyle{CodigoC}
   {basicstyle=\scriptsize,
	frame=single,
	language=C,
	numbers=left
   }
   
\lstdefinestyle{CodigoC++}
   {basicstyle=\small,
	frame=single,
	backgroundcolor=\color{gray75},
	language=C++,
	numbers=left
   }

\lstdefinestyle{Python}
   {language=Python,    
   }
\makeatother   