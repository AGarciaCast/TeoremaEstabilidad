\chapter*{Resumen}
Una de las herramientas fundamentales del Análisis Topológico de Datos es el diagrama de persistencia asociado a una función real definida en un espacio topológico. En este trabajo de fin de grado se enuncia y demuestra el \emph{Teorema de estabilidad de los diagramas de persistencia para la distancia bottleneck} siguiendo \cite{Cohen-Steiner2007}. Este resultado garantiza la robustez de los diagramas de persistencia bajo ciertas hipótesis leves, es decir, ``pequeñas'' perturbaciones en las funciones, dan lugar a diagramas de persistencia ``cercanos''.  

Para ello, comenzaremos introduciendo las nociones topológicas básicas con el objetivo de hacer autocontenido este trabajo. Continuaremos enunciando y demostrando el \emph{Teorema de estabilidad}. Por último estudiaremos diversos algoritmos para el cálculo de las distancias \emph{Hausdorff} y \emph{bottleneck} con el fin de ilustrar con ejemplos concretos la estabilidad de los diagramas de persistencia, centrándonos filtraciones de complejos simpliciales asociadas a nubes de puntos con un cierto ruido.
%%--------------
\newpage
%%--------------

\chapter*{Abstract}
Persistence diagrams of real-valued functions on topological spaces are one of the most important tools of Topological Data Analysis. In this undergraduate thesis, the \emph{Stability Theorem of Persistence Diagrams for the bottleneck distance} is stated and proved following \cite{Cohen-Steiner2007}. This theorem guarantees the robustness of the persistence diagrams under mild assumptions, that is, that ``small'' changes in the functions result in ``close'' persistence diagrams.

To do this, we will begin by introducing the basic topological notions to make this work self-contained. We will continue to state and prove the Stability Theorem. Finally, we will study various algorithms for calculating the Hausdorff and bottleneck distances to illustrate with several examples the stability of the persistence diagrams, focusing on filtrations of simplicial complexes associated with point clouds with some noise.

%%%%%%%%%%%%%%%%%%%%%%%%%%%%%%%%%%%%%%%%%%%%%%%%%%%%%%%%%%%
%% Final del resumen. 
%%%%%%%%%%%%%%%%%%%%%%%%%%%%%%%%%%%%%%%%%%%%%%%%%%%%%%%%%%%