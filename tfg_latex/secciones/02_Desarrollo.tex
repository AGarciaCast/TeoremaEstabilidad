\chapter{Desarrollo}

\section{Conocimientos previos y definiciones}
En esta sección se mostrarán las principales nociones de Topología Computacional, que nos darán el contexto y conocimientos necesarios para poder comprender el Teorema de Estabilidad y ser capaces de abordar su demostración.

\subsection{Motivación}
La Topología se centra en en el estudio de las diversas propiedades de los espacios topológicos y las funciones continuas. Recordemos la definición formal de espacio topológico:

\begin{definition}
Sea X un conjunto y $\tau \subset \mathcal{P}(X)$ una colección de de X. Diremos que $\tau$ es una topología en X si satisface las siguientes propiedades:

\begin{enumerate}
\item $\emptyset$ y X son elementos de $\tau$.
\item Si \{$U_{\lambda}$\}$_{\lambda\in\Lambda}$ es una familia de elementos de $\tau$ entonces
\[
\bigcup_{\lambda\in\Lambda} U_{\lambda}
\]
es un elemento de $\tau$ (La unión arbitraria de elementos de $\tau$ pertenece $\tau$).
\item Si $U_{1}, ..., U_{n}$ son elementos de $\tau$, entonces 
\[
U_{1} \cap ... \cap U_{n}
\]
es un elemento de $\tau$ (La intersección finita de elementos de $\tau$ pertenece $\tau$).
\end{enumerate}
El par (X,$\tau$) se denomina \textbf{espacio topológico} y a los elementos de $\tau$ se denominan \textbf{abiertos} de X.
\end{definition}

Mientras que en el subcampo de la Topología Computacional veremos como podemos hacer uso de diversos algoritmos para poder estudiar las propiedades de los espacios topológicos y ser capaces de resolver problemas topológicos computacionalmente. Para ello lo primero que necesitamos es una manera de representación de nuestros espacios topológicos, manteniendo sus propiedades topológicas.

\subsection{Complejos Simpliciales}
Una de las formas de representar un espacio topológico es a través de la descomposición del mismo en piezas más sencillas. Una descomposición en un complejo si sus piezas son topologicamente simples y sus intersecciones son piezas de dimensión inferior del mismo tipo \cite{EH}. Dentro de los complejos se puede observar que hay una gran variedad de tipos, dandonos distintos grados de abstracción. Nosotros vamos a trabajar con los complejos simpliciales, ya que nos darán unas buenas capacidades de computación.

Los complejos simpliciales los podemos estudiar desde un enfoque geométrico y desde un enfoque combinatorio. Partiremos de la definición de complejo simplicial desde el punto de vista geométrico.

















