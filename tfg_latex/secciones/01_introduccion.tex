\chapter{Introducción}
%%---------------------------------------------------------
LaSSSS introducción del TFG debe servir para que los profesores que evalúan el Trabajo puedan comprender el contexto en el que se realiza el mismo, y los objetivos que se plantean.

Esta plantilla muestra la estructura básica de la memoria final de TFG, así como algunas instrucciones de formato.

El esquema básico de una memoria final de TFG es el siguiente:
\begin{itemize}
\item[•] Resumen en español y inglés (máximo 2 páginas cada uno)
\item[•] Tabla de contenidos
\item[•] Introducción (con los objetivos del TFG)
\item[•] Desarrollo
\item[•] Resultados y conclusiones
\item[•] Bibliografía (publicaciones utilizadas en el estudio y desarrollo del trabajo)
\item[•] Anexos (opcional)
\end{itemize}

%%---------------------------------------------------------

\subsection{Motivación}
La Topología se centra en en el estudio de las diversas propiedades de los espacios topológicos y las funciones continuas. Mientras que en el subcampo de la Topología Computacional veremos como podemos hacer uso de diversos algoritmos para poder estudiar las propiedades de los espacios topológicos y ser capaces de resolver problemas topológicos computacionalmente. Para ello lo primero que necesitamos es una manera de representación de nuestros espacios topológicos, manteniendo sus propiedades topológicas.