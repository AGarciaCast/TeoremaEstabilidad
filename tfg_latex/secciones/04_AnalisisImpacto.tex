\chapter{Análisis de impacto}

Personalmente, este estudio me ha aportado de gran ayuda para poder ampliar mi conocimiento sobre el área de la topología computacional y sobre todo poder introducirme en cuestiones de topología algebraica más avanzadas a las vistas en la asignatura de Topología, que tienen muchas aplicaciones en diversas áreas del conocimiento. Además, me ha permitido adentrarme en el estudio de cuestiones teóricas relacionadas con el mundo del Data Science. Por tanto, este trabajo de fin de grado me ha aportado un gran crecimiento académico y vocacional.

Considero que este trabajo puede tener un impacto cultural, en la medida que se está exponiendo y recopilando información de diversas fuentes de forma que se explique de manera elemental conocimientos matemáticos, que pueden resultar difíciles de encontrar y comprender desde la fuente original. Además, esta colección de explicaciones parte del nivel de matemáticas de un alumno del Grado de Matemáticas e Informática, de forma que cualquier persona con dicho nivel de conocimiento será capaz de poder comprender la importancia, utilidad y el contenido matemático que respalda a la robustez de la homología persistente.

Este trabajo no aporta directamente al sector empresarial, económico o medioambiental, debido a que los algoritmos e implementaciones propuestos no son \emph{State of art}. Sin embargo, puede dar a conocer que los métodos del análisis de datos topológicos son robustos y por tanto deben tenerse en consideración para diversos tipos de empresas, proyectos o investigaciones con los que se manejen gran cantidad de datos. Cabe destacar que los métodos de TDA tienen ciertas ventajas por encima de los métodos tradicionales del análisis de datos. En particular, que no dependen de la elección de una métrica y proporcionan estabilidad frente al ruido (como hemos visto en este trabajo) \cite{tdaIBM}.  

Se pueden encontrar aplicaciones del TDA en el campo de la biología computacional para el estudio genético y el estudio de la identificación de procesos patógenos \cite{biologia}; también en el campo de la neurociencia \cite{neuronas}, en la lingüística computacional \cite{linguistica}, o en el ámbito de la medicina, como puede ser en el estudio de la oncología al integrar resultados de TDA al ámbito del Machine Learning \cite{oncologia}. Estos son algunas de las muchas aplicaciones actuales del TDA y todas ellas se sustentan en el resultado enunciado y demostrado en este trabajo, ya que sin la robustez de los diagramas de persistencia todas estas aplicaciones podrían no ser consistentes con sus resultados si se incorporan perturbaciones en los datos de entrada.